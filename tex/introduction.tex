% !TeX root = ../main.tex
% -*- coding: utf-8 -*-
% !TeX root = ../main.tex
% -*- coding: utf-8 -*-

\chapter{绪论}
\label{chpt:introduction}

与我们的直觉相反,磁场不需要是在重子数化学势这样一个量级
就能在色超导体中产生显著的效应。
正如文中\cite{}所讨论的,色超导体可以具有各种尺度的特征,
并且不同的物理现象可以出现在与每个尺度相当的场强上。
特别地是在这个所谓的色味连锁相(color-flavor locked phase)中,利用超导间隙、带电胶子的迈斯纳质量、
重子数化学势这三个不同的尺度可以确定了产生不同效应所需的磁场值。
因此,足够强的场的存在可以改变致密物质相的性质,
这反过来可能导致可观测的特征的改变。

\section{常用内容}

\begin{itemize}
  \item 参考文献的录入请参考\ref{sec:relatedwork:ref};
  \item 图片插入参考\ref{sec:relatedwork:table};
  \item 分数和公式参考\ref{sec:relatedwork:equation};
  \item Latex绘图工具参考\ref{sec:method:tikz};
  \item 代码块参考\ref{sec:method:code};
\end{itemize} 